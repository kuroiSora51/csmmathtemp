\documentclass[12pt]{article}
\usepackage{graphicx,tikz,tikz-network,tkz-euclide} % Required for inserting images
\usepackage{amsfonts,amsmath,amssymb}
\title{Math Club Fourth Week Solutions}
\usetikzlibrary{positioning}
\definecolor{processblue}{cmyk}{0.96,0,0,0}

\parindent 0pt
\date{October 2024}

\newcounter{problem}
\setcounter{problem}{0} % Initialize the counter at 0

\newenvironment{problem}{%
    \stepcounter{problem}
    \noindent\textbf{Problem \theproblem:}%
}{%
    \par
}

\newenvironment{solution}{%
    \vspace{1em} % Add space before the solution
    \noindent\textbf{Solution:}%
}{%
    \par
}

\newcommand{\multChoice}[5]{%
    \begin{tabular}{l @{\hskip 1.5cm} l @{\hskip 1.5cm} l @{\hskip 1.5cm} l @{\hskip 1.5cm} l}
    A. #1 & B. #2 & C. #3 & D. #4 & E. #5
\end{tabular}
}

\begin{document}


\maketitle



\begin{problem}[C][1][AMATYC Fall 2018/3]
   % Counting ^ Student Math League
   A 40 yd. by 30 yd. garden was subdivided into 1200 squares, each with
   side length 1 yd. A post was placed at each corner of each square (only one
   post was placed on shared corners). A single section of fence of length 1
   yard was placed on each shared side and also along the outside border. Let
   $P$ be the number of posts used and $F$ the number of fence sections used. Find $P + F$. 
\end{problem}


\begin{solution}[D]
See that, for each row, we have 41 posts, and that we have 31 rows, making $P=31 \times 41$.
For the fence sections in vertical, we see that, between two rows, there are 41 fences. Since we have 30 pairs of consecutive posts, it adds up to $30 \times 41$. Similarly, we have $40 \times 31$ for the horizontal fences. Therefore, 
\[
P+F= 31 \times 41 + 30 \times 41 + 40 \times 31 = 3741.
\]
\end{solution}

\vskip 1cm

\begin{problem}[A][4][AMATYC Fall 2017/6]
   % Algebra ^ Trig ^ Student Math League
   Assume $\sin x + \cos x = 1/4$. \\
   What is the value of $\sin^3x + \cos^3x $ ? 
\end{problem}
\multChoice{5/26}{11/32}{31/64}{47/128}{59/256}

\begin{solution}[D]
Notice that 
\[
\sin^3x + \cos^3x = (\sin x + \cos x)(\sin^2x - \sin x \cos x + \cos^2x)
\]
\[
= \frac{1}{4} \left ( 1 - \sin x \cos x \right ).
\]
We must find $\sin x \cos x$, using that
\[
(\sin x + \cos x)^2 = \sin^2x + 2\sin x \cos x + \cos^2x.
\]
\[
\Rightarrow \frac{1}{16} - 1 = 2\sin x \cos x.
\]
\[
\Rightarrow -\frac{15}{32} = \sin x \cos x.
\]
Finally,
\[
\sin^3x + \cos^3x = \frac{1}{4} \left ( 1 + \frac{15}{32} \right ) = \frac{47}{128}.
\]
\end{solution}

\vskip 1cm

\begin{problem}[C][2][AMATYC Fall 2017/9]
   % Knights and Knaves ^ Discrete ^ Student Math League
   Three people $(X, Y, Z)$ are in a room with you. One is a knight (knights always tell the
   truth), one is a knave (knaves always lie), and the other is a spy (spies may either lie or tell the truth). $X$ says “I am not a spy.”, $Y$ says “$X$ is a knave.”, and $Z$ says “$Y$ is a spy.” Which
   of the following correctly identifies all three people? 
   \begin{center}
      \begin{tabular}{l @{\hskip 1.5cm} l @{\hskip 1.5cm} l @{\hskip 1.5cm} l @{\hskip 1.5cm} l}
         A. $X$ is the spy, $Y$ is the knight, $Z$ is the knave & 
         B. $X$ is the spy, $Y$ is the knave, $Z$ is the knight & 
         C. $X$ is the knight, $Y$ is the knave, $Z$ is the spy & 
         D. $X$ is the knight, $Y$ is the spy, $Z$ is the knave & 
         E. $X$ is the knave, $Y$ is the spy, $Z$ is the knight 
      \end{tabular}
      \end{center}
\end{problem}



\begin{solution}[C]
See that $X$ being a knave contradicts the first statement. \\
If $X$ is the spy, then the second statement is false, so $Y$ is the knave, but then $Z$ would be the knight, contradicting their statement $\rightarrow \leftarrow$. \\
Therefore, $X$ is the knight. \\
If $Y$ were the spy, then the last statement is true, making $Z$ the knave, which contradicts the setup $\rightarrow \leftarrow$. \\
Thus, $Y$ is the knave, implying $Z$ is the spy. 
\end{solution}

\begin{problem}[C][4][AMATYC Fall 2007/7]
   % Lattice points ^ Counting ^ Student Math League
   A lattice point is a point with both coordinates integers. How many lattice points
    lie on or inside the triangle with vertices (0, 0), (10, 0), and (0, 8)? 
\end{problem}
 \multChoice{51}{52}{53}{54}{55}
\begin{solution}[A]
   These points lie under the hypotenuse of the triangle, that is, the segment that is part of the line $y=8-4x/5$ where $x \in [0,10]$.\\ For a lattice point $(a,b)$ the reader may check that each choice of $a$ gives $ \lfloor 9-4a/5 \rfloor$ choices for $b$, evaluating all $a$ with $0 \leq a \leq 10$ gives \\
    $9+8+7+6+5+5+4+3+2+1+1 = 5 \times 9 + 5 + 1 = 51$ possiblities
\end{solution}

\vskip 1cm

\begin{problem}[A][3][AMATYC Fall 2010/11]
   % Algebra ^ Student Math League
   Let:
    $$f(x) = \ln \left ( x+\sqrt{1+x^2} \right )$$
    Find $f^{-1}(\ln7)$ 
\end{problem}

\begin{solution}[B]
    We are asked to find $\varphi$ such that $f(\varphi)=\ln 7$, that is:
    \begin{align*}
        &\ln 7 = \ln \left ( \varphi+\sqrt{1+\varphi^2} \right ) \\
        \iff& 7 - \varphi = \sqrt{1+\varphi^2} \\
        \Rightarrow& \varphi^2 - 14\varphi + 49 = \varphi^2 + 1 \\
        \iff& \varphi = \frac{49-1}{14} = \frac{24}{7}
    \end{align*}
\end{solution}

\begin{problem}[G][3][AMATYC Spring 2004/12]
   % Pythagorean Theorem ^ Geometry ^ Student Math League
   A circular table is pushed into a corner of a rectangular room so that it touches both walls. A point on the edge of the table between the two points of contact is two inches from one wall and 9 inches from the other wall. What is the radius of the table? 
\end{problem}
\multChoice{5''}{12''}{15''}{17''}{20''}
\begin{solution}[17]
   \begin{center}
      \begin{tikzpicture}
          % Every aspect of the figure can be altered through these definitions
          \def\radius{4} \def\X{0.35} \def\labelSpacing{1.1}
          \def\A{110} \def\B{210} %\def\C{350} \def\D{70}
          
          \tkzDefPoints{0/0/A, \radius/0/R, 4.25/4.25/B, 0/4.25/C, 4.25/0/O, 0.5/2/X, 0.5/0/X'} 
          \tkzDefPointBy[projection=onto A--C](X) \tkzGetPoint{R1}
          \tkzDefPointBy[projection=onto B--C](X) \tkzGetPoint{R2}
         
          
         % Drawing
         \tkzDrawSegments(X,R1 X,R2 A,O O,B O,X A,X)
         \tkzDrawCircle(O,A) % Draw the circumcircle
         \tkzDrawSegments[dash pattern=on 5pt off 5pt](X,X') 
         \tkzDrawLine[add = 0.9 and 0,dash pattern=on 5pt off 5pt](A,C)
         \tkzDrawLine[add = 0 and 0.9,dash pattern=on 5pt off 5pt](C,B)
         
          \tkzLabelPoints[below](A,O){A,O}
          \tkzLabelPoints[above right](B,X){B,X}
          \tkzLabelPoint[above](C){C}
          \tkzLabelPoint[below right](X'){X'}
          \tkzLabelSegment[below](X',O){$r-2$}
          \tkzLabelSegment[right](X,X'){$r-9$}
          \tkzLabelSegment[above](O,X){$r$}
          %\node at (\radius - 0.3,\radius / 2 - 1) [right] {$\omega$}; 
      \end{tikzpicture}
      \end{center}
      
    Let $X$ be the point that is 2 inches from one wall and 9 inches from the other wall.
    Let $A,B$ be the points where the rectangle is tangent to the circle and let $C$
    be the vertice of the rectangle between $A$ and $B$ \\
    Call $r$ to the radius of the circle, since $AO=OB=r$ and $\angle ACB = 90^\circ$ we
    must have $AOBC$ to be a square, so we can define $X'$ to be the projection of $X$ onto 
    AO, meaning $OX'=r-2$ and $XX'=r-9$. \\
    By the Pythagorean Theorem:
    \begin{align*}
        &(r-2)^2 + (r-9)^2 = r^2 \\
        \iff& r^2-22r+85=0 \\
        \iff& (r-5)(r-17)=0
    \end{align*}
    Since $XX'$ exists, we cannot have $r=5$, so we must have $r=17$
\end{solution}


\vskip 1cm
These problems relate to Calculus content, the Student Math League does not require it, that being said, you may try them as well if you want
\vskip 1cm

\begin{problem}[R][4]
   % Functionals ^ Derivatives
   Find all functions $f:\mathbb{R} \rightarrow \mathbb{R}$ such that for any real value $x$:
   $$ f(x) = f'(x) + f''(x) + f'''(x) + \ldots $$ 
\end{problem}

\begin{solution}
   By implicit differentiation we see that 
   $$f'(x) = f''(x) + f'''(x) + \ldots$$
   That is
   $$ f(x) = f'(x) + (f''(x) + f'''(x) + \ldots) = 2f'(x) $$
   Now let $g$ be a function such that $f(x) = e^{g(x)}$, thus 
   $$ f'(x) = g'(x)e^{g(x)}$$ 
   \newpage
   Finally 
   $$e^{g(x)} = f(x) = 2f'(x) = 2g'(x)e^{g(x)}$$
   \begin{align*}
       &\Rightarrow g'(x) = 1/2 \\
       &\Rightarrow g(x) = x/2 + C \\
       &\Rightarrow f(x) = e^{x/2+C}
   \end{align*}
   Note that $e^{x/2+C}$ and $ce^{x/2}$ are equivalent expressions, just let $c=e^C$ 
\end{solution}

\begin{problem}[R][5]
    % Tangent Line Trick ^ Inequalities ^ Derivatives
    Let $a,b,c$ be three positive real numbers such that $a+b+c=2$. Show that
    $$ a^2+b^2+c^2+1 \geq \frac{1}{a^2+1}+\frac{1}{b^2+1}+\frac{1}{c^2+1}  $$
    Hint: Define some function $f$ and try to show that $f$ is concave in the interval we want. \\
    A function $f$ is concave if its tangent line lies above $f$, it happens when $f''(x)\leq 0$
    \end{problem}

\begin{solution}
   Let $f(x) = \frac{1}{x^2+1} - x^2$, then
    $$f'(x) = \frac{-2x}{(x^2+1)^2} - 2x$$
    $$f''(x) = \frac{2(3x^2-1)}{(x^2+1)^3}-2$$
    
    Let's show that $f''(x)<0$ for $x<0$:
    \begin{align*}
        \frac{2(3x^2-1)}{(x^2+1)^3}-2 &< 0 \\
        \iff 3x^2-1 &< (x^2+1)^3
    \end{align*}
    \newpage
    It is clear that the Right Hand Side grows much faster, but a formal proof can be done by the AM-GM inequality:
    \begin{align*}
        (x^2+1)^3 + 1 + 1 &\geq 3\sqrt[3]{(1)(1)(x^2+1)^3} = 3x^2+3\\
        \iff (x^2+1)^3 &\geq 3x^2+1 > 3x^2-1
    \end{align*}
    Since $f$ is concave, its tangent line lies above $f$ for $x>0$, in particular for $x=1$:

    \begin{align*}
        f(x) &\leq f'(1)(x-1) + f(1) \\
        \iff f(x) &\leq 2-5x/2 \\
        \Rightarrow f(a)+f(b)+f(c) &\leq 6-5\frac{(a+b+c)}{2}=1
    \end{align*}
    $\Box$
\end{solution}

\end{document}
