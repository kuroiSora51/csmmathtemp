\documentclass[12pt]{article}
\usepackage{graphicx,setspace,tikz} % Required for inserting images
\usepackage{amsfonts,amssymb,amsmath,float}
\title{Math Club First Week}
\author{Edgar}

\usetikzlibrary {positioning}
%\usepackage {xcolor}
\definecolor {processblue}{cmyk}{0.96,0,0,0}

\parindent 0pt
\date{September 2024}

\newcounter{problem}
\setcounter{problem}{0} % Initialize the counter at 0

\newcommand{\problem}[1]{
    \stepcounter{problem}
    \noindent\textbf{Problem \theproblem:} #1
     \\ % Add space after the problem statement
}

\newcommand{\solution}[1]{
    \vspace{1em} % Add space before the solution
    \noindent\textbf{Solution:} #1
     % Add space after the solution
}

\setlength{\baselineskip}{5\baselineskip}

\begin{document}

\maketitle

 \begin{spacing}{1.3}

\section*{Problems and Solutions}

\begin{problem}[A][1]
   % Functionals ^ Student Math League
   If $g(x-1) = x^2+1$ for all real numbers x, find $g(2)$.
\end{problem}

\begin{solution}
   In this case, we can directly set $x=3$ \\
   This comes from the fact that we will get $g(x-1)$ , similarly we would set $x=20$ if we were told to find $g(21)$
   $$ x=3 \Rightarrow g(2) = g(3-1) = 3^2+1=10$$
   $$ \therefore g(2)=10$$
\end{solution}

\begin{problem}[A][2]
   % Telescopic Cancellation
   What is the value of $ (\log_{624}625)(\log_{623}624)\ldots(\log_67)(\log_56) $?
\end{problem}    

\begin{solution}[4]
   Notice that \\
      $$m\log_ab = \log_ab^m$$
      $$\log_aa^x = x$$
      \\
      This means, for example     
      \[
      \left( \log_{624} 625 \right) \left( \log_{623} 624 \right) = \log_{623} \left( 624^{\log_{624} 625} \right)
      \]
      which means
      $$(\log_{624}625)(\log_{623}624) = (\log_{623}625)$$
      Similarly:
      $$(\log_{623}625)(\log_{622}623) = (\log_{622}625)$$
      At each step, we remove one term and keep the first term with the number $625$, so at the end we will have:

      $$(\log_{624}625)(\log_{623}624)\ldots(\log_67)(\log_56) = (\log_5625)$$   
      Since $625=5^4$ , the answer is 4 \\ \\

\end{solution}


\begin{problem}[C][2]
   % MinMax ^ Counting ^ Student Math League
   A student committee must consist of two seniors and three juniors. Five seniors are able to
   serve on the committee. What is the least number of junior volunteers needed if the selectors
   want at least 600 different possible ways to pick the committee? \\
\end{problem}

\begin{solution}
   The way we select our seniors is independent to how we choose our juniors, we can directly take 2 seniors out of 5 in $\binom{5}{2}=10$ ways, then call $k$ the minimum amount of juniors that we need to reach at least 600 different possible ways, since we can select the juniors in $\binom{k}{3}$ ways, we must have:
   \begin{align*}
       \binom{5}{2}\binom{k}{3} &\geq 600 \\
       \iff 10 \times \frac{k!}{3!(k-3!)} &\geq 600 \\
       \iff \frac{k(k-1)(k-2)}{6} &\geq 60 \\
       \iff k(k-1)(k-2) &\geq 360
   \end{align*}
   This expression is always increasing, and since we get 336 for $k=8$ and 504 for $k=9$, we conclude that $k=9$ is our answer.
\end{solution}

\end{spacing}
\end{document}



\hspace{2pt}