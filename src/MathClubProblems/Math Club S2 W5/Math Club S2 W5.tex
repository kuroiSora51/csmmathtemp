\input{Packages}
\input{Definitions}
\DeclareRobustCommand{\stirling}{\genfrac\{\}{0pt}{}}
%this removes numbering from the section, from some reason the \section*{} command doesnt work once you redefine it
\setcounter{secnumdepth}{0} 

% Define variables for week number and meeting date
\newcommand{\weekNum}{5} % Change this to update the week number
\newcommand{\meetingDate}{February 19th, 2025} 

\begin{document}
\pagestyle{empty}
\sloppy
\maketitle

\section{Selected Topic: Modular Arithmetic} %may change, but it'll be NT stuff

\begin{problem}[N][4][AMATYC Spring 2010/6]
    % Brute Force ^ Student Math League
    All solutions to the equation \( a^3 + b^3 + c^2 = 2010 \) (\( a, b, c \) positive integers) have the same value for \( a + b \). Find this value of \( a + b \).
\end{problem}
\multOpt[5]{$11$}[$12$][$13$][$14$][$15$]

\begin{problem}[N][4][AIME 1989/9]
    % MOD ^ AIME
    One of Euler's conjectures was disproved in the 1960s by three American mathematicians when they showed there was a positive integer $n$ such that
    $$ 133^5 + 110^5 + 84^5 + 27^5 = n^5 $$
    Find the value of $n$.
\end{problem}

\begin{problem}[N][4][Putnam 2024/A1]
    % Infinite Descent ^ MOD ^ Putnam
    Determine all positive integers $n$ for which there exist positive integers $a$, $b$, and $c$ satisfying
    \[
    2a^n + 3b^n = 4c^n.
    \]
\end{problem}

\begin{problem}[N][3][AoPS]
    % Chinese Remainder Theorem
    Show that for $c \in \mathbb{Z}$ and a prime $p$, exists an integer $x$ such that $p \mid  x^x - c$.
\end{problem}

\begin{problem}[N][4]
    % Primes ^ MOD 
    Find all pairs of prime numbers $(p,q)$ such that $2p^2 + 1 = q^5$.
\end{problem}

\begin{problem}[N][6][Justin Stevens]
    % Divisibility ^ GCD ^ p-adic valuation
    Let $a,b,c$ be positive integers, show that if $\text{lcm}(a,b,c) \cdot \gcd (a,b,c) = abc$, then $\gcd(a,b) = \gcd(b,c) = \gcd(c,a) = 1$
\end{problem}

\begin{problem}[N][5][USAMO 1973/2]
    % MOD ^ Algebra
    Let $\{ X_n \}$ be a sequence of integers defined by $(X_0, X_1) = (1,1)$, and $X_{n+1} = X_n + 2X_{n-1} , n \geq 1$. Let $\{Y_n\}$ be defined by $(Y_0, Y_1) = (1,7)$, and $Y_{n+1} = 2Y_n + 3Y_{n-1}, n \geq 1$. Thus the first few terms are
    \begin{align*}
        X : 1,1,3, 5, 11, 21, \ldots \\
        Y: 1, 7, 17, 55, 161, 487, \ldots
    \end{align*}
    Prove that no term greater than one occurs in both sequences.
\end{problem}

\begin{problem}[N][8][USAMO 1973]
   % Algebra ^ Divisibility ^ USAMO
   Show that the cube roots of three distinct prime numbers cannot be three terms (not necessarily consecutive) of an arithmetic progression.
\end{problem}

\begin{problem}[N][7][ISL 2005/6]
   % MOD
   Let $a,b$ be positive integers such that $b^n+n$ is a multiple of $a^n+n$ for all positive integers $n$. Prove that $a=b$.\\[2mm]
   \emph{Possible Hint:} Try and prove it for $a=1$. What happens if $a \equiv  b \pmod p$ for a prime $p$ large enough?
\end{problem}

\begin{problem}[N][5][Balkan]
    % MOD 
    Let $n$ be a positive integer with $n \geq 3$. Prove that 
    $$n^{n^n} - n^n$$
    is divisible by 1989.
\end{problem}

\begin{problem}[N][7][Putnam 2003/B3]
    % p-adic valuation ^ Putnam
    Show that for each positive integer $n$
    $$  n! = \prod_{i=1}^n \text{lcm} 
    \left \{ 1, 2, \ldots, \left \lfloor \frac{n}{i} \right \rfloor \right \}$$
    Here lcm denotes the least common multiple, and $\lfloor x \rfloor$ denotes the greatest integer $\leq x$.
\end{problem}

\begin{problem}[N][6][Putnam 2022/A3]
    % Counting ^ MOD ^ Putnam
    Let \( p \) be a prime number greater than 5. Let \( f(p) \) denote the number of infinite sequences \( a_1, a_2, a_3, \dots \) such that  \( a_n \in \{1,2,\dots, p-1\} \) and  \( a_n a_{n+2} \equiv 1 + a_{n+1} \pmod{p} \) for all \( n \geq 1 \). Prove that \( f(p) \) is congruent to 0 or 2 modulo 5.
    \end{problem}

\begin{solution}
    For any integer $a$ with $\gcd(a,p)=1$, we define $a^{-1} \pmod p$ as the unique integer $0 <x < p$ such that $a \cdot x \equiv 1 \pmod p$. We say that $x$ is the modular inverse of $a \mod p$. 
    We will show that $f(p) \equiv 0,2 \pmod{5}$ by showing that $f(p) = (p-2)(p-3)$.\\[2mm]
    Note that each sequence is uniquely determined by $a_1$ and $a_2$, for any $n \geq 3$, $a_n$ is given by the residue of $a_{n-2}^{-1} (1 + a_{n+1}) \pmod p$. Thus, we need to determine all the pairs $(a_1,a_2)$ that give valid sequences. Call $(a_1,a_2)$ a good pair if $1 \leq a_1, a_2 \leq p-2$ and $a_2 \not \equiv -a_1-1 \pmod p$.\\[2mm]
    \textbf{Claim 1 :} All valid sequences are formed by good pairs \\[1mm]
        Note that for any $i > 1$ we have $a_i \not \equiv -1 \pmod p$, otherwise the next term $a_{i+1}$ would be divisible by $p$,  and the sequence is defined in $[1,p-1]$. It is also true that $a_1 \not \equiv -1 \pmod p$, otherwise $-a_3 \equiv a_2 + 1 \Rightarrow a_2a_4 \equiv -a_2 \Rightarrow a_4 \equiv -1 \pmod p$. It follows that $1 \leq a_1, a_2 \leq p-2$. The additional value that $a_2$ cannot take is $-a_1 -1 \pmod{p}$. Indeed, that would cause $a_3 \equiv a_1^{-1}(a_2+1) \equiv -1 \pmod p$. $\Box$ \\[2mm]
    \textbf{Claim 2 :} All good pairs form a valid sequence \\[1mm]
        From Claim 1 we showed that being a good pair is a necessary condition, it turns out that it is also sufficient, we can select absolutely any pair $(a_1,a_2)$ considered a good pair, and it will work. We prove this by contradiction: \\
        Assume that exists a good pair $(a_1,a_2)$ that does not generate a valid sequence, that is, we still have an integer $k$ such that $a_k \equiv -1 \pmod p$ (equivalent to say $a_k = p-1$). Assume further that $k$ is the smallest integer with this property (the first term that invalidates the sequence). This means that $a_i \not \equiv -1 \pmod p$ for $1 \leq i < k$. We can check for small values and deduce that $k \geq 4$, then:
        \begin{align*}
            &a_ka_{k-2} \equiv a_{k-1} + 1 \pmod p \\
            \iff& -a_{k-2} \equiv a_{k-3}^{-1}(a_{k-2}+1) + 1 \pmod p\\ 
            \iff& 0 \equiv a_{k-3}^{-1} a_{k-2} + a_{k-3}^{-1} + a_{k-2} + 1 \pmod p  \\
            \iff& 0 \equiv(a_{k-2} + 1) (a_{k-3}^{-1} + 1) \pmod p
        \end{align*}
        This forces one of the terms $a_{k-2}, a_{k-3}$ to be $p-1$, contradicting the minimaility of $k$. $\Box$ \\
        Combining Claims 1 and 2, we conclude that a valid sequence is given by a good pair $(a_1,a_2)$, since we have $p-2$ choices for $a_1$ and $p-3$ choices for $a_2$. We conclude that $f(p) = (p-2)(p-3)$, the reader may check that this expression is always congruent with 0 or $2 \hspace{-4pt}\mod 5$ $\Box$

\end{solution}


% i may remove some of the problems i added if we have too many problems -- Edgar

% Honestly, I think these are too hard. Certainly, they are doable in a contest with lots of time and adequate preparation. I do not think they are suitable for a first introduction though. 

\end{document}