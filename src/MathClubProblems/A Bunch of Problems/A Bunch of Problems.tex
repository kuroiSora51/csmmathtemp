\input{Main/Packages}
\input{Main/Definitions}
\fancypagestyle{firstpageheader}{
    \fancyhf{}
% Header
    \fancyhead[L]{\ABunchOf{problems}}
    \fancyhead[R]{12/21/2024}
% Footer
    \fancyfoot[C]{\thepage}
    \renewcommand{\headrulewidth}{0.4pt} 
}
\definecolor{colordef}{cmyk}{0.81,0.62,0.00,0.22}
\definecolor{nicepurple}{RGB}{176, 92, 155}


\begin{document}
\sloppy
\maketitle

% Abstract 
\begin{center}
    %\textbf{\textcolor{colordef}{Abstract}} \\[0.5em]
    \begin{minipage}{0.8\textwidth} 
        \small 
        \setlength{\parindent}{0.3in}
        \lipsum[2] %i should change to actual abstract
    \end{minipage}
\end{center}

\vspace{2em} 


\setlength{\parindent}{0.3in} \lipsum[3] \\ %I will eventually put the actual text 
\setlength{\parindent}{0.3in} \lipsum[2] \\

\newpage
\thispagestyle{firstpageheader}

\section{Inequalities}

%{\textcolor{colordef}{This is some colored text.}}

\begin{problem}[Q][2][Pathfinder]
    % Inequalities ^ Algebra
    If $a,b,c$ are three sides of a triangle, and $a+b+c=2$, then prove that $a^2+b^2+c^2+2abc<2$
\end{problem}
\vskip 3mm

\begin{problem}[Q][4][Andreescu]
    %Inequalities ^ Algebra ^ Polynomials
    Let $P$ be a polynomial with positive real coefficients. Prove that if $P(1/x) \geq 1/P(x)$ holds for $x=1$, then it holds for all $x>0$
\end{problem}

\begin{problem}[Q][4]
    %Inequalities 
    Let $a,b,c$ be positive real numbers. Prove that $\frac{a^3}{b}+\frac{b^3}{c}+\frac{c^3}{a} \geq a^2+b^2+c^2$.
\end{problem}

\begin{problem}[Q][5][Pathfinder]
    %Inequalities 
    For $a,b \in \mathbb{R}^+$. Prove that 
    \begin{align*}
        \frac{a^3b}{(a+b)^4} \leq \frac{27}{256}
    \end{align*}
\end{problem}

\begin{problem}[Q][6][Lenin Cup 2024]
    %Inequalities ^ Algebra ^ Telescopic Cancellation
    Let $a_1,a_2,\ldots a_n$ be positive real numbers that verify $a_1a_2 \cdots a_n=1$. Show that 
    \begin{align*}
        \frac{a_1}{1+a_1}+\frac{a_2}{(1+a_1)(1+a_2)}+\ldots+\frac{a_n}{(1+a_1)(1+a_2) \ldots (1+a_n)} \geq \frac{2^n-1}{2^n}
    \end{align*}
\end{problem}

\begin{problem}[Q][12][Canadian Olympiad 2006]
    %Inequalities ^ MinMax ^ Jensen ^ Discrete 
    Consider a round-robin tournament with $2k + 1$ teams, where each team plays each other team exactly once. We say that three teams $X, Y$ and $Z$, form a cyclic triplet if $X$ beats $Y$ , $Y$ beats $Z$ and $Z$ beats $X$. There are no ties. Find the minimum and maximum possible number of cyclic triplet
\end{problem}


\newpage
\thispagestyle{firstpageheader}
\section{Number Theory}

\begin{problem}[N][1]
    % MOD ^ Primes 
    Find all pairs of prime numbers $(p,q)$ such that $2p^2+1=q^5$
\end{problem}

\begin{solution}
    aver
\end{solution}

\begin{problem}[N][4][AIME 1985]
    % Divisibility ^ GCD ^ AIME
    The numbers in the sequence $101, 104, 109,116, . . .$ are of the form $a_n = 100+n^2$, where $n = 1,2,3,\ldots$ For each $n$, let $d_n$ be the greatest common divisor of $a_n$ and $a_{n+1}$. Find the maximum value of $d_n$ as $n$ ranges through the positive integers.
\end{problem}


\begin{problem}[N][3][Iran 2005]
    % Divisibility ^ MOD ^ Primes
    Let $n,p>1$ be positive integers and $p$ be prime. Given that $n \mid p-1$ and $p \mid n^3-1$, prove that $4n-3$ is a perfect square.
\end{problem}

\begin{problem}[N][5][Putnam 2000]
    % Divisibility ^ Bezout's Identity ^ GCD
    Show that the expression 
    $$ \frac{\gcd(m,n)}{n} \binom{n}{m} $$
    is an integer for all pairs of integers $n \geq m \geq 1$
\end{problem}

\begin{problem}[N][4][Justin Stevens]
    % Divisibility 
    Prove that if $m,n \in Z^+: m \neq n$, then
    \begin{align*}
        \gcd \left(a^{2^m}+1, a^{2^n}+1\right) = 
        \begin{cases}
            1 & \text{if } a \text{ is even} \\
            2 & \text{if } a \text{ is odd}
        \end{cases}
    \end{align*}
\end{problem}

\begin{problem}[N][6][Polish 2003]
    % Polynomials ^ MOD ^ Primes
    Find all polynomials $W$ with integer coefficients satisfying the following condition: for every natural number $n$, $2^n-1$ is divisible by $W(n)$
\end{problem}

\begin{problem}[N][8][IMO 1990]
    % p-adic valuation ^ MOD ^ Primes
    Determine all integers $n>1$ such that
    $$\frac{2^n+1}{n^2} \in \mathbb{Z}$$
\end{problem}

\newpage

\section{Algebra}

\begin{problem}[A][4]
    % Algebra
    Let $a,b$ be real numbers such that  $a^2+b^2$, $a^3+b^3$ and $a^4+b^4$ are rational numbers. Show that $a+b$ and $ab$ are also rational numbers
\end{problem}

\begin{problem}[A][9][USAMO 2007]
    % Induction ^ Algebra
    Prove that for every nonnegative integer $n$, the number $7^{7^n}+1$ is the product of at least $2n + 3$ (not necessarily distinct) primes.
\end{problem}



\newpage

\section{Calculus and Analysis}


\end{document}
%rolan-rolan






